%%%%% PACKAGES %%%%%%
\usepackage[utf8]{inputenc}
\usepackage{nomencl}
\usepackage{fixltx2e}						% Suscript i text
\usepackage{lastpage}						% Giver mulighed for "side x af y"
\usepackage{url}							% Giver mulighed for links i fodnote
\usepackage{array}
\usepackage{calc}
\usepackage{pgf}
\usepackage{mwe}	
\usepackage[nomessages]{fp}
\usepackage{cite}
\usepackage{booktabs} 						% Punktopstilling til mufs bachelor
\usepackage{appendix}
\usepackage{lmodern}

%%%%% TIKZ %%%%%
\usepackage{tikz}
\usetikzlibrary{circuits.ee.IEC,shapes,arrows,spy,positioning,snakes,plotmarks,backgrounds}
\usepackage[american voltages, american currents,siunitx,smartlabels]{circuitikz}
\usepackage{pgfplots}

%%%%% OVERSAETTELSE OG TEGNSAETNING %%%%%
%\usepackage[danish]{babel}					% Dokumentets sprog
\usepackage[T1]{fontenc}					% Output-indkodning af tegnsaet (T1)
\usepackage{ragged2e,anyfontsize}			% Justering af elementer
\usepackage{fixltx2e}						% Retter forskellige fejl i LaTeX-kernen
																					
%%%%% FIGURES AND TABLES %%%%%
\usepackage{pbox}							% Tvungne linjeskift i tabeller
\usepackage{tabularx}						% Tvungne linjeskift i tabeller
\usepackage{graphicx} 						% Haandtering af eksterne billeder (JPG, PNG, EPS, PDF)
\usepackage{eso-pic}						% Tilfoej billedekommandoer paa hver side
\usepackage{multirow}                		% Fletning af raekker og kolonner (\multicolumn og \multirow)
\usepackage{multicol}         	        	% Muliggoer output i spalter
\usepackage{rotating}						% Rotation af tekst med \begin{sideways}...\end{sideways}
\usepackage{colortbl} 						% Farver i tabeller (fx \columncolor og \rowcolor)
\usepackage{xcolor}							% Definer farver med \definecolor
\usepackage{flafter}						% Soerger for at floats ikke optraeder i teksten foer deres reference
\let\newfloat\relax 						% Justering mellem float-pakken og memoir
\usepackage{float}							% Muliggoer eksakt placering af floats, f.eks. \begin{figure}[H]
%\usepackage{subfig}
\usepackage{caption}
\usepackage{subcaption}
\graphicspath{{./figures/}}					% Saetter default grafik-sti
\usepackage{placeins}						% \FloatBarrier
\usepackage{wrapfig}


%%%%% MATEMATIK MED MERE %%%%%
\usepackage{amsmath,amssymb,stmaryrd} 		% Avancerede matematik-udvidelser
\usepackage{mathtools}						% Andre matematik- og tegnudvidelser
\usepackage{textcomp}                 		% Symbol-udvidelser
\usepackage{rsphrase}						% Kemi-pakke til RS-saetninger,
\usepackage[version=3]{mhchem} 				% Kemi-pakke til flot og let notation af formler
\usepackage{siunitx}						% Flot og konsistent praesentation af tal og enheder med \si{enhed} og \SI{tal}{enhed}
%\sisetup{locale=DE}						% Opsaetning af \SI (DE for komma som decimalseparator) 
\sisetup{output-decimal-marker = {.}}		% Bruges ved engelske rapporter

%%%%% REFERENCES AND SOURCES %%%%%
\usepackage[danish]{varioref}				% Muliggoer bl.a. krydshenvisninger med sidetal (\vref)
\usepackage[numbers]{natbib}				% Udvidelse med naturvidenskabelige citationsmodeller
\usepackage{xr}								% Referencer til eksternt dokument med \externaldocument{<NAVN>}
\usepackage{glossaries}						% Terminologi- eller symbolliste (se mere i Daleifs Latex-bog)

%%%%% MISC. %%%%%%
\usepackage[pagecontinue=false]{pageslts}
\usepackage{titlesec}
\usepackage{epigraph}
\usepackage{lipsum}							% Dummy text \lipsum[..]
\usepackage[shortlabels]{enumitem}			% Muliggoer enkelt konfiguration af lister
\usepackage{pdfpages}						% Goer det muligt at inkludere pdf-dokumenter med kommandoen \includepdf[pages={x-y}]{fil.pdf}	
\pdfoptionpdfminorversion=6					% Muliggoer inkludering af pdf dokumenter, af version 1.6 og hoejere
\pretolerance=2500 							% Justering af afstand mellem ord (hoejt tal, mindre orddeling og mere luft mellem ord)
\usepackage[footnote,draft,english,silent,nomargin]{fixme}		

%%%%% MARGINS %%%%%%
\setlrmarginsandblock{3.5cm}{2.5cm}{*}		% \setlrmarginsandblock{Indbinding}{Kant}{Ratio}
\setulmarginsandblock{2.5cm}{3.0cm}{*}		% \setulmarginsandblock{Top}{Bund}{Ratio}
\checkandfixthelayout 						% Oversaetter vaerdier til brug for andre pakker

%%%%% AFSNITSFORMATERING %%%%%
\setlength{\parindent}{0mm}           		% Stoerrelse af indryk
\setlength{\parskip}{3mm}          			% Afstand mellem afsnit ved brug af double Enter
\linespread{1,1}							% Linie afstand
\titlespacing\section{0pt}{12pt plus 4pt minus 2pt}{0pt plus 2pt minus 2pt}
\titlespacing\subsection{0pt}{12pt plus 4pt minus 2pt}{0pt plus 2pt minus 2pt}
\titlespacing\subsubsection{0pt}{12pt plus 4pt minus 2pt}{0pt plus 2pt minus 2pt}

%%%%% BIBLIOGRAPHY %%%%%
%\bibpunct[,]{[}{]}{;}{a}{,}{,} 			% Definerer de 6 parametre ved Harvard henvisning
%\bibliographystyle{bibtex/harvard}			% Udseende af litteraturlisten.
\bibliographystyle{unsrtnat}



%%%%% TABLE OF CONTENTS %%%%%
\setsecnumdepth{subsection}		 			% Dybden af nummerede overkrifter 
\maxsecnumdepth{subsubsection}					% Dokumentklassens graense for nummereringsdybde
\settocdepth{subsubsection} 						% Dybden af indholdsfortegnelsen

%%%%% LISTS (ENUMERATE, ITEMIZE) %%%%%
\setlist{
  topsep=0pt,								% Vertikal afstand mellem tekst og listen
  itemsep=-1ex}								% Vertikal afstand mellem items
 
%%%%% HYPHENATION %%%%%
\usepackage{microtype}
%\renewcommand{\danishhyphenmins}{22} 		% bedre orddeling 
 
%%%%% "FIGUR" COMMAND %%%%%
\newcommand{\figur}[3]{
	\begin{figure}[htbp] \centering
		\includegraphics[width=#1\textwidth]{#2}
		\caption{#3}\label{fig:#2}
	\end{figure}} 
 
%%%%% "FIGURCROP" COMMAND %%%%%
\newcommand{\figurcrop}[8]{
	\begin{figure}[htbp] \centering
		\includegraphics[width=#1\textwidth, clip=true ,trim= #8cm #7cm #6cm #5cm]{#2}
		\caption{#3}\label{#4} % LEFT , BOTTOM , RIGHT , TOP
	\end{figure}} 

%%%%% VISUAL REFERENCES %%%%%
\usepackage[colorlinks]{hyperref}			% Danner klikbare referencer (hyperlinks) i dokumentet.
\hypersetup{colorlinks = true,				% Opsaetning af farvede hyperlinks (interne links, citeringer og URL)
   linkcolor = black,
   citecolor = black,
   urlcolor = blue}
\usepackage{bookmark}
\bookmarksetup{
  numbered,
  open,
  openlevel=0}								% Sætter dybden på ToC når pdf åbnes

%%%%% CAPTION SETTINGS %%%%%
\captionnamefont{\small\bfseries\itshape}	% Opsaetning af tekstdelen ('Figur' eller 'Tabel')
\captiontitlefont{\small}					% Opsaetning af nummerering
\captiondelim{. }							% Seperator mellem nummerering og figurtekst
\hangcaption								% Venstrejusterer flere-liniers figurtekst under hinanden
\captionwidth{\linewidth}					% Bredden af figurteksten
\setlength{\belowcaptionskip}{10pt}			% Afstand under figurteksten

%%%%% NAMING %%%%%
%\addto\captionsdanish{
%	\renewcommand\appendixname{Appendix}
	\renewcommand\contentsname{\scshape{Table of Contents}}
	\renewcommand\bibname{\scshape{References}}		
%	\renewcommand\appendixpagename{Appendix}
%	\renewcommand\appendixtocname{Appendix}
%	\renewcommand\cftchaptername{\chaptername~}		% Skriver "Kapitel" foran kapitlerne i ToC
%	\renewcommand\cftappendixname{\appendixname~}	% Skriver "Appendiks" foran appendiks i ToC
%}

%%%%% BEGIN CHAPTERSTYLE DESIGN %%%%%
\definecolor{numbercolor}{RGB}{153,25,21}		% Definerer en farve til brug til kapiteludseende
\newif\ifchapternonum

\makechapterstyle{OZTOPRAK-CHAPTER}{
  \renewcommand\beforechapskip{0pt}
  \renewcommand\printchaptername{}
  \renewcommand\printchapternum{}
  \renewcommand\printchapternonum{\chapternonumtrue}
  \renewcommand\chaptitlefont{\fontfamily{pbk}\fontseries{db}\fontshape{n}\fontsize{20}{35}\selectfont\raggedright\color{numbercolor}}
  \renewcommand\chapnumfont{\fontfamily{pbk}\fontseries{m}\fontshape{n}\fontsize{0.5in}{0in}\selectfont\color{numbercolor}}  
  \renewcommand\printchaptertitle[1]{%
    \noindent
    \ifchapternonum
    \begin{tabularx}{\textwidth}{X}
        \vspace{-2.5cm}
    	{\let\\\newline\chaptitlefont\scshape{##1}\par} 
    \end{tabularx}
    \vspace{-2cm}
    \par\vskip-2.5mm\hrule
    \else
    \vspace{-3cm}
    \begin{tabularx}{\textwidth}{Xl}
    	\raisebox{-0pt}{\chapnumfont \thechapter} & {\hspace{1cm} \parbox[b]{\linewidth}{\chaptitlefont \scshape{##1}}}
    \end{tabularx}
    \par\vskip1mm\hrule
    \fi
  }
}
%%%%% END CHAPTERSTYLE DESIGN %%%%%

%%%%% BEGIN PAGESTYLE DESIGN %%%%%
\makepagestyle{OZTOPRAK-PAGE}
\makepsmarks{OZTOPRAK-PAGE}{%
	\createmark{chapter}{left}{shownumber}{}{. \ }
	\createmark{section}{right}{shownumber}{}{. \ }
	\createplainmark{toc}{both}{\contentsname}
	\createplainmark{lof}{both}{\listfigurename}
	\createplainmark{lot}{both}{\listtablename}
	\createplainmark{bib}{both}{\bibname}
	\createplainmark{index}{both}{\indexname}
	%\createplainmark{glossary}{both}{\glossaryname}
}
\nouppercaseheads												% Ingen Caps oenskes

\makeevenhead{OZTOPRAK-PAGE}{\leftmark}{}{}				% {Name}{L}{C}{R}
\makeoddhead{OZTOPRAK-PAGE}{}{}{ {\leftmark}  }	        % {Name}{L}{C}{R}
%\pagenumbering{roman}
\makeevenfoot{OZTOPRAK-PAGE}{}{Page \thepage \space of \lastpageofpart}{}			% {Name}{L}{C}{R}
\makeoddfoot{OZTOPRAK-PAGE}{}{Page \thepage \space of \lastpageofpart}{}		    %{Name}{L}{C}{R}

\makeheadrule{OZTOPRAK-PAGE}{\textwidth}{0.5pt}						% Tilfoejer en streg under sidehovedets indhold
\makefootrule{OZTOPRAK-PAGE}{\textwidth}{0.5pt}{1mm}					% Tilfoejer en streg under sidefodens indhold

\copypagestyle{AAUchap}{OZTOPRAK-PAGE}								% Sidehoved for kapitelsider defineres som standardsider, men med blank sidehoved
\makeoddhead{AAUchap}{}{}{}
\makeevenhead{AAUchap}{}{}{}
\makeheadrule{AAUchap}{\textwidth}{0pt}
\aliaspagestyle{chapter}{AAUchap}				% Den ny style vaelges til at gaelde for chapters
%%%%% END PAGESTYLE DESIGN %%%%%						
													
%%%%% VALG AF CHAPTER OG PAGESTYLE %%%%%									
\chapterstyle{OZTOPRAK-CHAPTER}						% Valg af kapiteludseende				
\pagestyle{OZTOPRAK-PAGE}							% Valg af sidehoved og sidefod

%%%%% PART SETTINGS %%%%%
\aliaspagestyle{part}{empty}					% Fjerner sidetal på part sider
\renewcommand{\afterpartskip}{\vfil}

%%%%% INFO DECLARATIONS %%%%%%
\newcommand{\lastpageofpart}{\lastpageref{pagesLTS.roman}}
\newcommand*{\supervisor}[1]{\def\supname{#1}}
\newcommand*{\thesistitle}[1]{\def\ttitle{#1}}
\newcommand*{\subtitle}[1]{\def\stitle{#1}}
\newcommand*{\examiner}[1]{\def\examname{#1}}
\newcommand*{\degree}[1]{\def\degreename{#1}}
\renewcommand*{\author}[1]{\def\authorname{#1}}
\newcommand*{\addresses}[1]{\def\addressname{#1}}
\newcommand*{\university}[1]{\def\univname{#1}}
\newcommand*{\department}[1]{\def\deptname{#1}}
\newcommand*{\group}[1]{\def\groupname{#1}}
\newcommand*{\faculty}[1]{\def\facname{#1}}
\newcommand*{\subject}[1]{\def\subjectname{#1}}
\newcommand*{\keywords}[1]{\def\keywordnames{#1}}
\newcommand*{\Authorcontact}[1]{\def\Authorcontactinformation{#1}}
\newcommand*{\Authorname}[1]{\def\Authornameinformation{#1}}
\newcommand*{\Authoremail}[1]{\def\Authoremailinformation{#1}}

%--------------------------------
% code in latex
\usepackage{listings}
\usepackage{color}

\definecolor{mygreen}{RGB}{28,172,0} % color values Red, Green, Blue
\definecolor{mylilas}{RGB}{170,55,241}

\definecolor{dkgreen}{rgb}{0,0.6,0}
\definecolor{gray}{rgb}{0.5,0.5,0.5}
\definecolor{mauve}{rgb}{0.58,0,0.82}

\lstset{frame=tb,
  language=python,
  aboveskip=3mm,
  belowskip=3mm,
  showstringspaces=false,
  columns=flexible,
  basicstyle={\small\ttfamily},
  numbers=none,
  numberstyle=\tiny\color{gray},
  keywordstyle=\color{blue},
  commentstyle=\color{dkgreen},
  stringstyle=\color{mauve},
  breaklines=true,
  breakatwhitespace=true,
  tabsize=3
}

%\usepackage[utf8]{inputenc}
\usepackage{listingsutf8}
\lstset{language=Matlab,%
    %basicstyle=\color{red},
    breaklines=true,
    extendedchars=true,
    morekeywords={matlab2tikz},
    keywordstyle=\color{blue},%
    morekeywords=[2]{1}, keywordstyle=[2]{\color{black}},
    identifierstyle=\color{black},%
    stringstyle=\color{mylilas},
    commentstyle=\color{mygreen},%
    showstringspaces=false,%without this there will be a symbol in the places where there is a space
    numbers=left,%
    numberstyle={\tiny \color{black}},% size of the numbers
    numbersep=9pt, % this defines how far the numbers are from the text
    emph=[1]{for,end,break},emphstyle=[1]\color{blue}, %some words to emphasise
    %emph=[2]{word1,word2}, emphstyle=[2]{style},    
}